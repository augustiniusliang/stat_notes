\documentclass[lang=cn,device = normal,mode = fancy,black]{elegantbook}
\usepackage[switch,mathlines]{lineno}
\usepackage{tikz-cd}
\usepackage{braket}
\setcounter{tocdepth}{3}
\newcommand{\tr}{\text{tr}}
\newcommand{\bigzero}{\mbox{\normalfont\Large\bfseries 0}}
\newcommand{\rvline}{\hspace*{-\arraycolsep}\vline\hspace*{-\arraycolsep}}
\newcommand{\thermfrac}[3]{\left( \frac{\partial #1}{\partial #2} \right)_{#3}}
\newcommand{\jacobian}[4]{\frac{\partial \left(#1, #2\right)}{\partial \left(#3, #4\right)}}
\newcommand{\ppp}[2]{\frac{\partial #1}{\partial #2}}
\linenumbers
\title{量子统计力学笔记}
\author{梁莫言}
\institute{清华大学}
\date{\today}

\begin{document}
\frontmatter
\nolinenumbers

\begin{titlepage}
    \vspace{70cm}
    \begin{center}
        \begin{flushright}
        {\Large \heiti 纯粹量子统计力学笔记} \\
        {\large Notes on Pure Quantum Statistical Physics} \\
        {\kaishu 梁莫言}
        \vline
    \end{flushright}
    \vspace{20pt}
    \begin{figure*}[htbp!]
        \centering
        \includegraphics[scale=0.4]{graphics/960px-Landau.jpg}\\
        {\large {Lev D. Landau} }
    \end{figure*}
    \end{center}
    
    \newpage\pagestyle{empty}
    \topskip0pt
\vspace*{\fill}
\textsc{Ubi eras quando ponebam fundamenta terræ ; 
indica mihi, si habes intelligentiam.
Quis posuit mensuras ejus, si nosti ; 
vel quis tetendit super eam lineam ; Super quo bases illius solidatæ sunt ; 
aut quis demisit lapidem angularem ejus, cum me laudarent simul astra matutina,
et jubilarent omnes filii Dei ; Quis conclusit ostiis mare,
quando erumpebat quasi de vulva procedens;cum ponerem nubem vestimentum ejus,
et caligine illud quasi pannis infantiæ obvolverem ; 
Circumdedi illud terminis meis,
et posui vectem et ostia,
et dixi: Usque huc venies, et non procedes amplius,
et hic confringes tumentes fluctus tuos.
Numquid post ortum tuum præcepisti diluculo,
et ostendisti auroræ locum suum ; 
Et tenuisti concutiens extrema terræ,
et excussisti impios ex ea ; 
Restituetur ut lutum signaculum,
et stabit sicut vestimentum:
auferetur ab impiis lux sua,
et brachium excelsum confringetur.
Numquid ingressus es profunda maris,
et in novissimis abyssi deambulasti ; 
Numquid apertæ sunt tibi portæ mortis,
et ostia tenebrosa vidisti ; 
Numquid considerasti latitudinem terræ ; 
indica mihi, si nosti, omnia!}

\vspace*{\fill}
\end{titlepage}
\tableofcontents
\mainmatter
\linenumbers
\resetlinenumber
\part{统计物理的基础}
\chapter{量子统计物理学的假设和基本原理}
量子统计力学所研究的问题是\textbf{大量}服从 Schrodinger 方程的粒子的 \textbf{平衡态统计行为}。对于经典粒子来说,在粒子数变多时,解出这些粒子所服从的 Hamilton 方程已经不可能,找出满足它们的初值条件的特解更是不可能。在量子的情形下,要考虑这些粒子的相互作用下再解出 Schrodinger 方程,为人来说已经不可能。但是,粒子的大量性可以使我们做出一些遍历性的假设,进而发现这些粒子所满足的统计规律,这些规律不依赖于所要求的 Schrodinger 方程的初值条件和通解。注意此处的\textbf{统计} 不是对于每一个粒子求统计平均,而是对于大量服从同样的 Schrodinger 方程的系统求统计平均。因此很必要先对这个概念进行厘清。为此,先介绍量子统计中的重要工具:密度算符。本章的目标就是计算各种情形下的密度算符。
\section{量子力学的密度矩阵表述}
\subsection{密度矩阵的定义和性质}
为了描述统计理论所要求的\textbf{按照不同概率出现的量子态,} 我们定义下述密度算符:
\begin{linenomath}\begin{equation}
\hat{\rho} := \sum_{\text{可能的量子态}\psi} p_\psi\ket{\psi}\bra{\psi}.
\end{equation}\end{linenomath}
其中,$p_\psi$ 是态 $\psi$ 在量子态空间中被找到的概率(严格定义将在下一章给出)。注意,这里不是说量子态是某态有多大概率,而是一个\textbf{统计平均}。这样,我们能描述的态就不只局限于能用\textbf{波函数} 表出,而是可以构造出一些混合在一起但不相干的态,我们将前者称作 \textbf{纯态},而把后者称为 \textbf{混合态}. 

我们看到,因为概率是归一的,所以我们有
\[
\tr \hat{\rho} = \sum_i\sum_{\text{可能的量子态}\psi} p_\psi \braket{i|\psi}\braket{\psi|i}  = \sum_{\text{可能的量子态}\psi} p_\psi \braket{\psi|\psi} = 1.
\]
其中用到了\(\sum \ket{i}\bra{i} = \hat{I}.\) 所以,密度矩阵的迹是归一的。
那么,如何判断一个密度矩阵所描述的是纯态还是混合态?我们可以看算符 $\hat{\rho}^2$ 的迹,
\[
\tr \hat{\rho}^2 = \sum_{\text{可能的量子态}\psi} p_\psi^2 \leq \sum_{\text{可能的量子态}\psi} p_\psi = 1.
\]
当且仅当是纯态时取等,在混合态的情形则是严格小于 1. 注意,在纯态情形,还成立 $\hat{\rho}^2 = \hat{\rho}.$

当然,密度矩阵不必在本征态表象中写出,譬如我们便可以在坐标表象中写出矩阵的形式,只需要写出:
\[
\begin{aligned}
    \hat{\rho} &= \int d^3\vec{q} d^3\vec{q'}\sum_{\text{可能的量子态} \psi} p_\psi \ket{\vec{q}}\braket{\vec{q}|\psi}\braket{\psi|\vec{q'}}\bra{\vec{q'}}\\
    &= \sum_{\text{可能的量子态} \psi} p_\psi \int \psi^*(\vec{q'})\psi(\vec{q}) d^3\vec{q} d^3\vec{q'}\ket{\vec{q}}\bra{\vec{q'}},
\end{aligned}
\]
那么,矩阵元就是
\[
\rho(\vec{q},\vec{q'}) = \sum_{\text{可能的量子态} \psi} p_\psi \psi^*(\vec{q'})\psi(\vec{q}).
\]

\subsection{部分迹与相干性}
\label{partial}
我们可以看到,若考虑了一个系统的全部自由度,那么它(理论上)便可以表示为 Schrodinger 方程的解,因此它便是一个纯态。所以,一个物理态可以被标为纯态,就意味着我们取得了对于这个系统的所有自由度的信息。但是,若我们只能获知其中一些自由度,而另一些自由度我们无从知道,那么我们便无法解出 Schrodinger 方程。
\textbf{密度矩阵处理的就是信息缺失时的量子态计算。} 为了从所有自由度中取出一些自由度,而将剩下自由度“遮住”,我们可以引入下述运算:

对于一个系统,可以分出它们的子系统 $A$ 和 $B,$ 它们分担了整个系统的自由度。于是我们便可以把算符 Banach 空间写成直积 $L=L_A\otimes L_B,$ 其中 $L_A$ 和 $L_B$ 分别对应了他们对应的自由度。

那么系统的 Hamilton 矩阵可以写为
\[
\begin{pmatrix}
  H_A
  & \rvline & H_{int} \\
\hline
  H_{int}^\dagger & \rvline &
  H_B
\end{pmatrix},
\]
其中$H_A$ 和 $H_B$ 是两个子系统自身的 Hamilton 子矩阵,而 $H_{int}$ 是相互作用。同样,密度算符的矩阵可以写作
\[
\begin{pmatrix}
  \rho_A
  & \rvline & \rho_{ent} \\
\hline
  \rho_{ent}^\dagger & \rvline &
  \rho_B
\end{pmatrix},
\]
同样,非对角项描写的是子系统 $A$ 和 $B$ 之间的\textbf{量子相干}。但若我们只知道且只能测量子系统 $A$ 的自由度,我们可以定义以下的\textbf{部分迹}:
\[
\tr_B: L\to L_A, \hat{M} \mapsto \sum_{i} \braket{i|_B\hat{M}_B|i}.
\]
对于密度矩阵取迹,我们便得到了一个矩阵,这个矩阵只关涉子系统 $A$ 的自由度,这样得到的矩阵就是我们现在能够描述的关于这个量子系统的全部信息。部分迹操作不仅是数学上的简化,它在物理上代表了由于环境耦合导致的\textbf{退相干过程}。为了详细理解这一点,让我们计算两个二能级的粒子,他们是纠缠的:
\[
\ket\Psi = \alpha\ket{00}+\beta\ket{11}
\]
其中参数 $\alpha,\beta$ 满足归一化关系 $|\alpha|^2+|\beta|^2 = 1.$
由这个纯态可以计算出密度矩阵:
\[
 \hat{\rho }_{AB} = \left( \begin{matrix} {\left| \alpha \right| }^{2} & 0 & 0 & \alpha {\beta }^{ * } \\  0 & 0 & 0 & 0 \\  0 & 0 & 0 & 0 \\  {\alpha }^{ * }\beta & 0 & 0 & {\left| \beta \right| }^{2} \end{matrix}\right) 
\]
可以看到,相干性体现在非对角项上。但当我们实行部分迹运算:
\[ {\rho }_{A} = {\operatorname{tr}}_{B}\left( {\rho }_{AB}\right)  = \langle {\left. 0\right| }_{B}{\rho }_{AB}\left| {0{\rangle }_{B}+\langle 1{\left. \right| }_{B}{\rho }_{AB}}\right| 1{\rangle }_{B}  = \operatorname{diag}\left( |\alpha|^2 , |\beta|^2\right).\]
这样,量子态就“退相干”了。这种大量态的平均引起的退相干称为\textbf{热化},这就是数学上部分迹引起的物理结果。这意味着子系统 $A$ 已经失去了量子相干性,它不再能发生干涉,其行为完全等同于一个“以 $|\alpha|^2$ 概率处于 $0$ 态,以 $|\beta|^2$ 概率处于 $1$ 态”的\textbf{经典随机系统}。这个事实有一个统计解释:当我们对于大量这样的系统进行平均时,因为环境(子系统 $B$) 的状态我们未加测量,所以我们观测到的子系统状态 $A$ 就是非相干的。这种解释被称作\textbf{系综解释},这些大量相同系统所组成的集合被称为\textbf{系综}。因此我们对于任意算符$\hat O,$ 在系综下都可以进行平均:
\[
\braket{O}_{e} := \sum_{\psi} p_\psi \braket{\psi|\hat{O}|\psi} = \tr\left( \hat{O}\hat{\rho} \right).
\]
\subsection{Liouville 定理}
若在加和中的每一态都是 Schrodinger 方程的解,也就是服从下述方程:
\[
i\hbar\frac{\partial}{\partial t}\psi = \hat H \psi,
\]
那么我们可以计算
\[
\begin{aligned}
    i\hbar \frac{\partial}{\partial t} \hat{\rho} &= i\hbar \frac{\partial}{\partial t} \sum_{\text{可能的量子态}\psi} p_\psi\ket{\psi}\bra{\psi} \\
    &= i\hbar\sum_{\text{可能的量子态}\psi} p_\psi\frac{\partial}{\partial t}(\ket{\psi}\bra{\psi}) \\
    &= \sum_{\text{可能的量子态}\psi} p_\psi \left( \hat{H}\ket{\psi}\bra{\psi}-\ket{\psi}\bra{\psi}\hat{H} \right) \\
    &= \hat{H}\hat{\rho}-\hat{\rho}\hat{H}.
\end{aligned}
\]
这就是 Liouville 定理,它给出了密度矩阵的运动方程:
\begin{linenomath}\begin{equation}
\boxed{i\hbar \frac{\partial}{\partial t} \hat{\rho} = \left[ \hat{H},\hat{\rho} \right]}.
\end{equation}\end{linenomath}
于是,只要密度算符不显性含时间,那么它就是运动积分。 注意,只要这些态是 Schrodinger 方程的本征态,那么密度矩阵就满足这个方程。我们看到,平衡态统计物理所研究的是平衡态的情形,于是密度算符对于时间的偏导必为 0,因而它是一个运动积分。但是在一般的多体系统中,守恒量有且仅有 $\text{能量} H, \text{总动量} \vec{P}, \text{总角动量}\vec{M}.$ 所以我们断定,
\begin{linenomath}\begin{equation}
\hat{\rho} = \hat{\rho}(\hat{H},\hat{\vec{P}},\hat{\vec{M}}).
\end{equation}\end{linenomath}
我们看到,$\hat\rho$ 对于近独立的各个子系统是可乘的,因此$\ln\hat\rho$ 是可加的。因此
\begin{linenomath}\begin{equation}
\ln\hat\rho = \alpha + \beta \hat H + \vec\gamma \cdot\vec P + \vec\delta\cdot \vec M.
\end{equation}\end{linenomath}
对于每一个子系统成立。
\section{统计物理的对象与特点}

\subsection{宏观物体和子系统}
\noindent\textit{
本部分可以参考 {Landau \& Lifschitz, para. 1.}}

统计物理所关心的是大量微观粒子所形成的宏观物体,这些宏观物体内部粒子数众多,这些粒子不仅给物体本身的哈密顿量带来很多自由度,而且和环境也有各式各样的相互作用。因此按照 \ref{partial} 节所述,我们应该将宏观物体和环境的整体看作一个整体,由一个 Hamiltonian $H=H_{A} + H_{\text{环境}} + H_{int}$ 支配。所以我们在取子系统时,应将宏观物体取为关心的子系统,而把环境自由度通过部分迹操作去掉。这样得到了约化的密度矩阵 $\rho.$ 按照我们的讨论,现在物体的量子态是退相干的混合态。因此对于任意一个本征态(在上述近似的意义下),我们有概率\[w_n=w(E_n)=\rho_{nn}.\]

为了研究这个密度矩阵,我们写出原来(未约化)的哈密顿量的 Liouville 方程:
\[
i\hbar \frac{\partial}{\partial t} \hat{\rho}_A = \left[ \hat{H}_A,\hat{\rho}_{A} \right] + \left( \hat{H}_{int} \hat{\rho}_{ent}^\dagger - \hat{\rho}_{ent} \hat{H}_{int} ^\dagger\right).
\]
现在我们需要作出一个核心假设:因为外界与子系统之间的相互作用那样复杂,又那样随机,我们按照对称性便不能期望在远长于弛豫时间的平均下,这些相互作用的均值比起我们关心的物体自身的各观测量是可观测的。在数值上,这意味着:
\[\langle H_{int} \rangle \ll H_A, \langle \rho_{ent} \rangle \ll \rho_A.\]
在这个近似的意义上,我们便可以说约化后的 Hamilton 量和密度矩阵也满足 Liouville 定理。事实上,这个假设也是我们可以给子系统 $A$ 定义(近似守恒)能量的前提。

\subsection{系综平均与各态历经假设}
我们的实验测量的事实上是一个算符 $\hat{O}$ 的时间平均值,这个时间间隔对我们来说足够短,但比起微观的粒子跃迁频率的倒数又十分长:
\[
\langle \hat{O} \rangle_T = \frac{1}{T} \int_0^T \langle \Psi(t) | \hat{O} | \Psi(t) \rangle dt.
\]
现在引入一个重要的假设:尽管 $H_{int}$ 相比起粒子自身的哈密顿量来说是很小的,但这一微扰的存在使得粒子并不能处于一个定态,而是在不同的态之间跃迁。具体来说,我们对于态之间的跃迁有 Fermi 黄金准则:
系统从初态 $|i\rangle$ 跃迁到末态 $|f\rangle$ 的单位时间概率 $W_{i \to f}$ 为
\[W_{i \to f} = \frac{2\pi}{\hbar} |\langle f | \hat{H}_{int} | i \rangle|^2 \delta(E_f - E_i),\]
这意味着对于宏观物体的可观测状态 $E,$ 在微观层面上却在能壳 $[E-\Delta E/2,E+\Delta E/2]$ 间非常多不同的微观状态上不断地跃迁,这使我们可以假定在足够长的 $T$ 时间内,已经等概率地走遍了能壳内(在这里就是系综内)所有的量子态。在数学上,这就是\textbf{取部分迹}的操作。所以我们便可以引入这样的\textbf{各态历经假设}:
\begin{linenomath}\begin{equation}
\langle \hat{O} \rangle_T \to \sum_{\Psi} p_\Psi \langle \Psi(t) | \hat{O} | \Psi(t) \rangle = \sum_{\Psi}\tr\left( \hat\rho\hat O \right) = \braket{\hat O}_e, T\to+\infty.
\end{equation}\end{linenomath}
这就是说,只要$T$ (严格地说,比起跃迁频率的倒数) 趋于无穷大,那么时间平均等于系综平均(大概是在 $\frac{1}{TH_{int}^2}$ 的量级上)。然而需要指出,就像 Landau 也曾指出的那样,按照不确定原理,宏观物体达到定态需要跃迁时间趋于无穷,而这是不可能的。但是在一个范围内,我们可以近似认为这物体处于定态,因此 Liouville 定理给出密度矩阵的近似不变性,当然是在给定的尺度上。一个简单的数值估计可以看出,这不仅是可能的,还是几乎自然就实现的。

近年来,研究指出近于孤立的量子系统也显出统计系统的特性,也就是尽管没有 $H_{int},$ 量子系统也或许是各态历经的。这种各态历经性的根据在于当粒子数充分大时,量子混沌现象和随机矩阵具有关联。研究者们提出了所谓本征态热化假说(Eigenstate Thermalization Hypothesis),解释了统计性的纯粹量子起源。
\label{time}
\section{一般统计分布}
\subsection{固定子系统能量的情形:微正则分布}
我们先来研究固定能量的情形。但按照测量的不确定性原理,以及相互作用带来的能量的小展宽,同时考虑到取数学面所可能带来的奇异性,我们不能够直接将能量表象下的分布函数写成 Dirac $\delta$ 函数,而是应该取一个小展宽 $\Delta E,$ 这个展宽应该比微观能级的间隙大(但是数量级相当),但在宏观上可以当成一个微元。
还需要考虑,按照上节的论述,在这一能壳中的量子态是等可能遍历的。这样我们就可以写出这种情况下的密度矩阵:
\begin{linenomath}\begin{equation}
\hat{\rho}\left( E \right) = \frac{1}{\Delta \Gamma \left( E \right)} \sum_{E_n\in\left[ E-\frac{\Delta E}{2}, E+\frac{\Delta E}{2}\right]}\ket{n}\bra{n}.
\label{microcan}\end{equation}\end{linenomath}
这种系综和它导出的分布叫做\textbf{微正则系综}和\textbf{微正则分布},其中 $\Delta \Gamma\left( E \right)$ 是在 $\left[ E-\frac{\Delta E}{2}, E+\frac{\Delta E}{2}\right]$ 能壳内的量子态数,或正像 Landau 称呼的那样,它叫作能量宏观态 $E$ 的 \textbf{统计权重}。通过引入小于等于 $E$ 的态数 $\Gamma \left( E \right),$ 我们可以写:
\[
\Delta \Gamma \left( E \right) \sim \frac{\partial\Gamma}{\partial E}\left( E \right)\Delta E := \Omega \left( E \right)\Delta E.
\]
对于能够分成很多个子系统的能量固定的大系统来说,我们看到,这时密度矩阵就是:
\begin{linenomath}
  \begin{equation}
    \hat\rho = \frac{1}{\Delta\Gamma\left( E \right)}\sum_{E_i} \delta(\sum_{i} E_i - E)\prod_i \Delta\Gamma^{\left( i \right)}(E_i) {\hat\rho}^{\left( i \right)}\left( E_i \right),
  \label{subsys}\end{equation}
\end{linenomath}

其中,对于密度矩阵的乘法是直积。在这里使用 Dirac $\delta$ 函数是恰当的,因为各个子系统的密度矩阵中已经自然包含展宽。如果我们不想关心能壳上的细致态分布,我们便可以将这个矩阵如同在 \ref{partial} 节中所讨论的那样,对于每一个子矩阵取部分迹,这时我们所得到的就是在能量表象下的密度矩阵:
\[
\hat{\tilde{\rho}}(\left[ E_i \right]) = \frac{1}{\Delta\Gamma\left( E \right)}\sum_{E_i} \delta(\sum_{i} E_i - E)\prod_i \Delta\Gamma^{\left( i \right)}(E_i),
\]
其中已经用到密度矩阵的迹是 $1.$ 如果我们将带 $\Delta$ 的符号全部换成微元号 $d$ (同样,这间隔比微观能级的间隙大,但在宏观上可以当成一个微元),便得到了按能量的概率分布:
\begin{linenomath}\begin{equation}
\hat{\tilde{\rho}}(\left[ E_i \right]) = \operatorname{const.} \sum_{E_i} \delta(\sum_{i} E_i - E)\prod_i d\Gamma^{\left( i \right)}(E_i).
\end{equation}\end{linenomath}

\subsection{熵与温度}
\label{entropy}
子系统的\textbf{熵}定义为统计权重的对数: \[S(E):=k\ln\left( \Delta\Gamma\left( E \right) \right) = -k\ln\left( w(E) \right) = -\sum_n kw(E_n)\ln(w(E_n)).\]
最后一个式子(称为熵的\textbf{ Gibbs 定义式})是合理的,因为这里定义的熵是一个宏观量,也就是一个时间平均,关于这个时间的的论述可以见\ref{time}节。$k$ 是用来规定量纲的常数。用更一般的密度矩阵语言,我们可以写出\[S=-k\tr(\hat\rho\ln\hat\rho).\] 显然可以看到,S的大小表征了能壳内量子态的多少,也就是多少表征了粒子运动的“混乱程度”。

我们定义\textbf{绝对温度}是\[T = \frac{1}{\frac{\partial{S}}{\partial{E}}},\] 因此我们见到温度的量纲是 $[E][k]^{-1}.$ 显然,熵 $S$ 、能量 $E$ 及温度 $T$ 在其中一者固定时,另外两者以这样或那样的函数形式互相依赖(或不变);若其中二者固定,第三者也不变。

这样,借助熵,我们就可以把密度矩阵重写成:
\begin{linenomath}\begin{equation}
\hat\rho = \frac{1}{\Delta\Gamma\left( E \right)}\sum_{E_i} \delta(\sum_{i} E_i - E)\prod_i e^{\frac{S_i}{k}} {\hat\rho}^{\left( i \right)}\left( E_i \right)=\operatorname{const.} \sum_{E_i} e^{\frac Sk} \delta(\sum_{i} E_i - E)\prod_i {\hat\rho}^{\left( i \right)}\left( E_i \right).
\end{equation}\end{linenomath}

若我们像上节那样,取这个密度矩阵关于所有子系统同一能量下的量子态的部分迹,我们便遇到一个困难:因为所有的密度矩阵都将会变成 $1,$ 从而无法计量能量的分布密度。为了解决这个困难,我们可以注意到
\[
d\Gamma_i=\prod_i\frac{d \Gamma_i}{d E_i}dE_i \sim \prod_i\frac{\Delta \Gamma_i}{\Delta E_i}d E_i = \operatorname{const.} e^{\frac Sk} \prod_i d E_i.
\]
因为比起随着状态而急遽变化的熵,能壳的宽度可以看作大致是一个常数 (或者认为这个能壳宽度已被吸收进熵的表达式里)。这样,我们就有:
\begin{linenomath}\begin{equation}
dw = \operatorname{const.} e^{\frac Sk} \delta(\sum_{i} E_i - E) \prod_i d E_i.
\label{dist}
\end{equation}\end{linenomath}
其中前面的常数由约束 \[\int dw = 1\] 确定。

最后,跟随 Landau,我们指出熵在微观上,表征了能级平均距离的负对数。这是自然的:因为能级越近,在能壳内便可能存在更多的态。
\subsection{熵增定律}
在统计中,我们研究的是最盖然情形,在这里就对应了统计权重最大的的情形。我们考虑多个子系统的分布,由 (\ref{dist}) 式可以看到,在一个封闭系统(或可以分为许多小子系统的封闭系统)中,最盖然的情形就是熵最大的情形。这使得我们按照概率论中类似的假设(比如,大数定律)可以引入下述假设:因为系统倾向于自发地从盖然的情形演化到更盖然的情形,所以系统的熵自发是不减少的。这就是\textbf{熵增定律,} 注意熵增定律并不依赖于系统的具体形式。

但是,这里有一个矛盾:对于经典系统中,相空间运动方程总是时间反演对称的,这导致表观上所有遵循经典方程(或近似遵循它们)的粒子的运动都应该是可逆的,这与熵增定律构成一个矛盾。对此,我解释如下:我们所考虑的是大量相点的运动的规律,在很多相点以一种方向性运动,但另外少数相点以相反的方式运动时,我们在统计上就看到了一种方向性,尽管这两种方式都是力学上可能的。在这里所看到的,就是这种情形。
\subsection{固定子系统温度的情形:Gibbs 分布}
现在我们来看温度固定(换句话说,能量不固定)的情形。这里最方便的方式是将一个能量确定的大系统(因此微正则系综)中分出一个小系统 $A$,将剩下的看作环境(因为这里是固定温度的情形,所以我们将这个环境称为\textbf{热库})。一个可能的质疑是:这里是否满足我们所说的条件 $H_{int} \ll H_A?$ 可做一个估计:对于线度 $L,$ 离子浓度 $n$ 的宏观物体,可以有 $H_{int} \sim n L^2 \delta, H_A \sim L^3 n,$ 其中 $\delta$ 是粒子的线度;我们现在有$H_{int}/H_A=\delta/L,$ 这就是说:在我们研究的情况下,不论如何划分系统,$H_{int} \ll H_A$ 几乎总是成立。

\begin{remark}
  现代纳米尺度物理和强耦合统计物理研究的内容,就是当 $H_{int} \ll H_A$ 不成立时的新奇物理现象。当子系统非常小(例如一个分子与热库接触)时,我们还是可以通过重整化 Hamilton 量的方式来使统计分布具有原来的形式,但是这是讨论能级已经失去意义了。例如,量子随机游走模型就是这种模型之一例。
\end{remark}

于是,我们现在承认这样的近似,并且对于环境量子态所有能量求和,并对对应的直和分量求部分迹。
由 (\ref{subsys}) 式我们有:
\[
  \hat\rho = \frac{1}{\Delta\Gamma\left( E \right)}\sum_{E_i} \Delta\Gamma^{\left( A \right)}(E_i)\Delta\Gamma^{\prime}(E-E_i) .
\]

现在,让我们取某个特定的 $i,$ 这时就有 $\Delta\Gamma^{\left( A \right)}(E_i) = g(E_i),$ 其中 $g(E_i)$ 是能级 $E_i$ 的简并度,为了简便起见令它等于1. 并且
\[
 dw = \left[ \hat{\rho} \right]_{ii} \frac{1}{\Delta E} dE = \operatorname{const.} \frac{\Delta\Gamma^{\prime}(E-E_i)}{\Delta\Gamma\left( E \right)} dE_i.
\]  
因为取出的子系统相比起大热库是如此小,下述近似是容许的:
\[
  \frac{\Delta\Gamma^{\prime}(E-E_i)}{\Delta\Gamma\left( E \right)} \sim \frac{\Delta\Gamma(E-E_i)}{\Delta\Gamma\left( E \right)} = \exp\left( \frac1k\left( S(E-E_i)-S(E) \right) \right) = \exp\left( -\frac{E_i}{kT} \right),
\]
其中已经用到 $\frac{\partial S}{\partial E} = \frac{1}{T},$ 并考虑到 $E_i \leq E.$ 所以,
\begin{linenomath}\begin{equation}
\boxed{dw = A\exp\left( -\frac{E_i}{kT}\right) dE_i} ,
\end{equation}\end{linenomath}
若计及了简并度,那么就有 \[dw = Ag\left( E_i \right) \exp\left( -\frac{E_i}{kT} \right) dE_i.\]
其中 $A$ 是归一化常数。这就是由\textbf{正则系综}(也就是上面将物体与大热库接触的构造)导出的\textbf{正则分布},或 \textbf{Gibbs (1901) 分布}。这个式子从算符层面写出,就是:
\begin{linenomath}\begin{equation}
\hat{\rho} = \frac{1}{Z}\exp\left( -\frac{\hat{H}}{kT} \right).
\end{equation}\end{linenomath}

其中定义了配分函数 \[Z = \tr\left( \exp\left( -\frac{\hat{H}}{kT} \right) \right).\] 今后将会看到,配分函数蕴含了一个系统的全部统计信息。
\subsection{Gibbs 分布和熵最大原理 *无偏见原则}
从另一个方面也可以导出 Gibbs 分布,这种方面使人更加明了 Gibbs 分布满足熵最大的原理. 我们使用带约束的变分法,将熵作为 Lagrangian:
\[
L = -k\tr(\hat{\rho}\ln\hat{\rho}) + (-\alpha + 1) (\tr(\hat{\rho}) - 1) - \beta (\tr(\hat{\rho} \hat H) - E)
\]
将 $\hat{\rho}^\prime = \hat{\rho} + \delta\hat{\rho}$ 代入,求变分就得到:
\[
k\tr\left( \delta\hat{\rho}\left( \ln\left( \hat{\rho}  \right) + \alpha + \beta \hat{H} \right) \right) = 0.
\]
注意,引入 Lagrange 乘子后,变分就是任意的。所以我们有:
\[
k\ln\hat{\rho} = -\alpha - \beta \hat{H}.
\]
这就是 Gibbs 分布的形式。以后我们也会记 $\beta := (kT)^{-1},$ 称为逸度。

现在,让我们取二阶变分:
\[
\delta^2 L = -k\tr (\delta\hat \rho \delta\hat \rho \hat{\rho}^{-1}) \ll 0.
\]
其中用到了密度矩阵只有正数本征值。在Gibbs 分布的情形下,一个简单的估计可以看出在平衡位置附近,熵有一个尖锐的峰。这种方法直观地体现了为什么 Gibbs 分布是一个熵最大的分布。

熵最大原则贯穿了整个统计物理,这一点在统计学上也有根据。一个统计问题本质上就是只知道一个系统的平均能量 $\langle E \rangle$ 时,对它的分布的推测问题。然而,根据这个信息,理论上存在无数种概率分布 $\{p_i\}$ 都能满足这个平均值。我们有很多选择,但是按照我们之前的讨论,如果我们选择了一个熵较小的分布,这意味着我们未加证明地假设了某些态出现的概率比其他态更高,而这种倾向性并没有任何物理观测证据支持。 这种无根据的倾向性,就是“偏见”。统计力学中的‘平衡态’,本质上是最不具偏见的描述状态。
许多学生误以为平衡态是某种‘演化结果’。但从信息论视角看,平衡态其实是我们作为观测者,在已知有限宏观量的前提下,能对系统做出的最公正、最客观的推断。
任何熵更小的状态(非平衡态),都隐含了某种尚未消失的‘秩序’或‘关联’(偏见)。随着系统发生跃迁和退相干,这些残余的‘秩序’被环境抹除,我们被迫放弃这些‘偏见’,最终回归到那个最无知、但也最正确的最大熵分布。

\subsection{平均能量的计算}
我们已经有配分函数 \[Z=\sum_{s} g(E_s)\exp\left( -\beta{E_s} \right)\],我们现在就可以计算一些有关系统的参量,对于正则系综来说,这个量就是\textbf{平均能量。}
藉由系综平均的表达式,我们看到平均能量就是 \[\bar{E} = \tr\left( \hat H\hat \rho \right) = \frac{1}{Z}\sum_{s} E_sg(E_s)\exp\left( -\beta{E_s} \right) = -\frac{\partial Z}{Z\partial \beta}.\]

\section{全同粒子的统计}
{\noindent \textit{本节需要第 \ref{sec_quant} 章内容。}}

前述的统计中,大量粒子的统计使我们导出了 Gibbs 分布,这个分布的导出有一前提:态只和能量有关。这一前提在全同粒子的统计时被打破了,或者说:变得十分不方便,因为在这里粒子数成为标记本征态的唯一标记。同样值得注意的是,子系统不仅可以和外部环境交换能量,还可以交换粒子,这些粒子所带来的能量是无法确算的。更进一步,在容器内可能发生的化学反应或者物理-核物理过程也会带来新的粒子。所有这些都促使我们构造一种与粒子数直接相关的统计。我们要做的,就是和以前一样,对于除了粒子数之外的其他量子数求部分迹。
\subsection{粒子数与化学势}
我们从第 \ref{sec_quant} 章知道,对于全同粒子可以定义粒子数算符 $\hat{N} = a^\dagger a.$ 就和能量 $E$ 一样,我们知道在统计物理中算符 $\hat{N}$ 的定域本征值只有在一个尺度上是对的,因为对于墙壁,粒子总可以通过隧穿的方式进出。但是只要势垒的高度远高于粒子的能量,我们便可以认为这种近似是足够好的。同时我们要注意,这里要求在密度矩阵内关涉的态是 Hamilton 算符 $\hat H$ 和粒子数算符 $\hat N$ 的本征态。显然可以看出,在构造粒子的时候若采用哈密顿量的本征态,那么得到的波函数便是满足要求的。

现在系统变得与粒子数显性相关,这使我们仿照温度定义一个新的物理(宏观统计)量:化学势。它的定义如下:
\[\left( \frac{\partial S}{\partial N} \right)_E \equiv -\frac{\mu}{T}.\]
化学势描述了在能量不变的情况下,加入一个粒子对于熵(混乱度)的增幅。注意到,化学势和能量具有同一量纲。所以,直觉上,它是某种“势能”,这将在下一章讨论。

化学势 $\mu,$ 熵 $S$ 与 粒子数 $N$ 是三个变量,但只要其中两个固定( $E,T $ 也固定),那么第三个也不变。因此,我们可以写出微分式:
\begin{linenomath}\begin{equation}  dS = \frac{dE}{T} - \frac{\mu dN}{T}.\label{diff_chem}\end{equation}\end{linenomath}

\subsection{带有粒子数算符的微正则分布与 Gibbs 分布}
与能量不同,粒子数总是离散的,而且精确地以 $1$ 为单位。所以只要势垒足够高,粒子数的分布便可以很好地用 Dirac $\delta$ 函数来描述。于是我们可以类似 (\ref{microcan}) 写出带有粒子数的微正则分布:
\begin{linenomath}\begin{equation}
\hat{\rho}\left( E,N_0 \right) = \frac{1}{\Delta \Gamma(E,N_0)} \sum_{E_{(N,\sigma)}\sim E} \ket{N,\sigma}\bra{N,\sigma}\delta\left( N-N_0 \right).
\end{equation}\end{linenomath}
其中 $\sigma$ 代表除了粒子数之外的其他量子数,同时注意这里的 Dirac $\delta$ 函数本质上就是 Kronecker 符号。那么,对于多个子系统来说,我们自然有
\begin{linenomath}\begin{equation}
  \hat{\rho} = \frac{1}{\Delta\Gamma\left( E_0,N_0 \right)} \sum_{E_i,N_i} \delta\left( \sum E_i -E_0 \right)\delta\left(\sum N_i - N_0\right) \prod_i\Delta\Gamma(E_i,N_i) \hat{\rho}\left( E_i, N_i \right).
\end{equation}\end{linenomath}
我们现在将一个粒子数与能量守恒的系统分为一个小子系统和一个大\textbf{粒子库-热库}。现在我们对于粒子库-热库的能级求和,并对所有密度矩阵部分迹,就有:
\begin{linenomath}\begin{equation}
  dw(E_i,N_i) = \operatorname{const.} \frac{\Delta \Gamma'(E_0-E_i,N_0-N_i)}{\Delta \Gamma(E_0,N_0)} dE_i dN_i.
\end{equation}\end{linenomath}
在做和推导正则分布一样的近似,并注意到微分式 (\ref{diff_chem}),我们就有
\begin{linenomath}\begin{equation}
  dw(E_i,N_i) = A g(E_i,N_i)\exp\left( -\frac{E_i-\mu N_i}{kT} \right) dE_i dN_i,
\end{equation}\end{linenomath}
或写成算符形式
\begin{linenomath}\begin{equation}
  \hat\rho = \frac{1}{\Xi} \exp\left( -\frac{\hat H-\mu{\hat{N}}}{kT} \right).
\end{equation}\end{linenomath}
这就是\textbf{巨正则系综}推导出的\textbf{巨正则分布}。其中定义了巨配分函数:
\[
\Xi = \tr\left( \exp\left( -\frac{\hat H-\mu{\hat{N}}}{kT} \right) \right),
\]
这之中求和是对整个 Fock 空间。同样,巨配分函数也蕴含整个系统的所有统计信息。注意,这里$\rho$ 是否是运动积分取决于 $H$ 与 $N$ 是否对易。如果我们像我们一开始说的那样,对于一个与粒子数无关的系统作二次量子化,那么不会有任何问题。但我们看到,在 Hamilton 算符 $H$ 依赖于升降算符时,只要 Hamilton 量不会凭空产生或湮灭粒子(比如 Hamilton 算符中没有类似 $a^\dagger a^\dagger$ 的项),这样的条件也一定是成立的,而这是大多数情况。

正如在正则系综中,我们对于逸度求导可以得到平均能量一样,我们现在也可以通过对巨配分函数的求导得到巨正则分布的特征参量,就是\textbf{平均占有数。}具体来说,我们设参量 \[\alpha = \frac\mu T,\] 那么平均占有数就可以写成 \[ \bar N = \tr\left( \hat\rho \hat N \right) = \frac1\Xi \sum_iN_i \exp\left( \alpha N_i-\beta E_i \right) = \frac{\partial \Xi}{\Xi\partial \alpha}.\] 注意,平均能量的求法和正则系综一样。
\subsection{Fermi 分布和 Bose 分布}
对于全同的,各自具有能级 $\varepsilon_s$ 的不可分辨粒子,我们有占据数表象下的算符:
\[
\hat H = \sum_s \varepsilon_s \hat{n}_s,\quad  N = \sum_s \hat{n}_s.
\]
因此各个能级的密度矩阵是直积。我们想要求它的能级分布,则应该将它的粒子数取迹。因此对于 Bose 子,各个占有数可以是任意自然数,因此我们计算巨配分函数:
\[
\Xi_s = \sum_{n_s} \left[ \exp\left( -\frac{\varepsilon_s-\mu}{kT} \right)\right]^{n_s} = \frac{1}{\exp\frac{\mu-E_s}{kT}-1}
\]
因此 $\varepsilon_s$ 能级上的占有数期望值就是
\begin{linenomath}\begin{equation}
\boxed{\bar{n}_s =\frac{\partial \Xi_s}{\Xi_s\partial \alpha}= -\frac{1}{\exp\frac{\mu-E_s}{kT}-1}}. \label{bose}
\end{equation}\end{linenomath}
这叫做 \textbf{Bose-Einstein 分布。}

而对于 Fermi 子,容许的占有数是 $0$ 和 $1.$ 于是巨配分函数就是 \[\Xi_s = \left( 1+\exp\left( \frac{\mu-E_s}{kT} \right) \right), \]而占有数期望值是:
\begin{linenomath}\begin{equation}
\boxed{\bar{n}_s =\frac{\partial \Xi_s}{\Xi_s\partial \alpha}= \frac{1}{\exp\frac{\mu-E_s}{kT}+1}}. \label{fermi}
\end{equation}\end{linenomath}
这叫做 \textbf{Fermi-Dirac 分布。}










\chapter{宏观热力学量与 Gibbs 分布}
“宏观\textbf{热力学量}” 事实上是一个同语反复(\textit{*tautologos}),因为只有宏观量才能为我们所测量。本章的目标是:介绍常见的热力学量,并指出他们与 Gibbs 分布的参量之间的关系。我们已经知道了更一般的情况,就是粒子数可以变化(由化学势支配)的情况,让我们把这些情况也考虑在内。

{\noindent \kaishu 本节采取略差异于俄派的记法,对于欧美派的记法比较可见下表.}
\begin{table}[htbp!]
    \centering
    \begin{tabular}{lll}
        \hline 
        本稿记法及命名 & Landau \& Lifschitz & Kardar \\
        \hline 
        $E$ 内能 & $E$ 内能 & $U$ 内能 \\  
        $W$ 焓 & $W$ 焓 & $H$ 焓 \\
        $F$ 自由能 & $F$ 自由能 & $F$ Helmholtz 自由能 \\
        $\Phi$ Gibbs 函数 & $\Phi$ 热力学势 & $G$ Gibbs 自由能 \\
        $\Omega$ 热力学势 & $\Omega$ “另一种”热力学势 & $\Omega$ 巨势 \\
        \hline \hline
    \end{tabular}
    \caption{记法及命名对照表}
\end{table}
\section{内能}
我们已经知道,在系统与外界不发生宏观的相互作用时,我们可以写出下式:
\[
dE = -TdS + \sum_i\mu_idN_i.
\]
上式中已经计及多种粒子所具有的不同化学势。其中 $E$ 不是指物体的全部能量,而是在一个物体既不平动,也不转动的参考系中,系统的能量。(在转动参考系中,等效势能计入 $E.$) 这里的假设是物体的运动是经典的。
这个式子的物理意义是:当没有宏观的相互作用时,物体与外界仅可通过传热和交换粒子传递相互作用。
\subsection{准静态、可逆过程与元功}
\label{work}
现在想要知道当外界条件发生变化时,物体的内能将会发生怎样的变化。
为了不破坏平衡态,我们认为引入相互作用的过程是如此的慢,以至于它在力学上可以使用准静态过程(对于力学量 $\lambda$ 来说,这要求 $\frac{d\lambda}{dt}\ll 1$ )描述。但是若我们在引入相互作用时使熵 $S$ 和粒子数 $N$ 发生了改变,那么内能的形式将会难以写出。所以在这里我们只研究粒子数不变的可逆过程( $S$ 不变)。

在引入相互作用时,我们可以将熵随时间的变化 $d_t S$ 展成力学参量变化 $d_t \lambda$ 的级数,并假定这是引起熵变化的唯一来源(稍后将解释其中原因)。 由于平衡态的假设和熵增原理,我们可以知道领头阶的项是二阶的。于是:
\[
\frac{d S}{d t} = A \left( \frac{d \lambda}{d t} \right)^2.
\]
这就得到
\[
\frac{d S}{d \lambda} = A \frac{d \lambda}{d t}.
\]
这意味着只要 $d_t \lambda$ 趋于零,$S$ 就与 $\lambda$ 无关。这就是说,在一般情况下,当 $\lambda$ 的变化不引起其他使熵增大的改变(没有耗散或泄流),那么\textbf{可逆过程和准静态过程是等价的。} 在这种情况下,因为没有热化的传递,我们也把这种过程称为\textbf{绝热过程}。

在可逆性的保证下,我们来研究外界对于宏观物体的作用。物体受到的作用用物体对外所做的 \textbf{元功} 表示,元功就是微小的功。我们将元功写成
\[
dW = \sum_{i } dW_i = \sum_i Y_idy_i,
\]
其中 $y_i$ 是表示物体与外界作用的\textbf{广义坐标}(一般要求可加),而 $Y_i$ 是与广义坐标共轭的\textbf{广义力}。因为功具有能量的量纲,这对于互为共轭的力和坐标也有量纲上的约束。这样我们就可以写出内能微元的一般表达式:
\begin{linenomath}\begin{equation}
dE= -\sum_{i} Y_idy_i + TdS + \sum_j \mu_jdN_j. \label{internal_energy}
\end{equation}\end{linenomath}

现在我们来证明一个重要的结论,就是平均能量对于广义坐标的求导就是可逆过程下物体所受的广义力。这一事实的需要用到经典力学中的一个结论:对于能量 $E(p,q,\lambda),$ 有 \[
\frac{dE(p,q,\lambda)}{dt} = \frac{\partial E(p,q,\lambda)}{\partial t} +\left\{ E,E \right\} = \frac{\partial E(p,q,\lambda)}{\partial t},
\]
其中大括号表示经典 Poisson 括号。而因为只有 $\lambda$ 显含时间,我们便可在微观上写出
\[
\frac{dE}{dt} = \frac{\partial \bar E}{\partial \lambda} \frac{d \lambda}{dt}.
\]
但是在宏观上
\[
\frac{dE}{dt} = \left( \frac{\partial \bar E}{\partial \lambda} \right)_{S,\left\{ N_j \right\}} \frac{d \lambda}{dt},
\]
括号外的下标表示保持写出的参量不变。综合两式,我们就有
\begin{linenomath}\begin{equation}\frac{\partial \bar E}{\partial \lambda} = \left( \frac{\partial  E}{\partial \lambda} \right)_{S,\left\{ N_j \right\}}.\label{macro_micro1}\end{equation}\end{linenomath}
这样,我们便可以由统计计算出相应的参量,并和实验结果进行比较,这样就在微观与宏观之间建立联系。最后指出,宏观参量的改变导致微观能级的改变,从而将二者联系起来是可能的。
\subsection{压强}
与体积 $V$ 共轭的广义力被称为压强$p:$ \[p = -\left( \frac{\partial E}{\partial V} \right)_{S,\left\{ N_j \right\}}. \] 为了说明压强的物理意义,让我们只改变一个微元 $\delta \sigma$ 附近的柱状空间的体积,设我们改变了 $dl$ (向外为正)。  这时元功可以写成:
\[dW = -p\delta \sigma dl.\] 又因为由力学知道 \[dW = -\delta F dl,\] 上式中 $F$ 向外为正。因此有 \[p = \frac{\delta F}{\delta \sigma},\] 就是说,\textbf{压强就是单位面积受的压力。}

\subsection{内能对应的 Maxwell 关系}
我们看到,内能满足了下述微分等式:
\[
dE = -pdV + TdS + \mu dN
\]
对此,我们现在假设 $dN = 0.$ 我们对于一个全微分 $dE$ 应该有 Cauchy-Riemann 条件
\begin{linenomath}\begin{equation}
\left( \frac{\partial T}{\partial V} \right)_{S,N} = -\left( \frac{\partial p}{\partial S} \right)_{V,N}.
\end{equation}\end{linenomath}
这个式子称为 \textbf{Maxwell 关系,} 它是内能作为能量所必然推出的,具有普适性。这就是说, $(p,V,T,S)$ 不是独立的变量,其中只有两个是独立的。将来我们要定义许多能量,每一个能量按照自己的全微分式都有自己的 Maxwell 关系。对于粒子数还有两个 Maxwell 关系:
\[
\left( \frac{\partial \mu}{\partial S} \right)_{V,N} = \left( \frac{\partial T}{\partial N} \right)_{V,S}
\]
和
\[
\left( \frac{\partial \mu}{\partial V} \right)_{S,N} = -\left( \frac{\partial p}{\partial N} \right)_{V,S}.
\]

\subsection{热容与熵\quad 宏观物体的平衡条件}
我们已经知道,系统与外界的相互作用除了通过力学与交换粒子的方式,尚可以通过一种以熵为“广义坐标”(如果这么说不是对于这个词的滥用的话)的方式来传递“混乱度”,也就是所谓热化(见 \ref{entropy} 节),那么我们就将其称为传热,传热的表达式常用 $dQ = TdS$ 表示。为了讨论传热和温度的关系,我们可以定义下列热容:
\[
C_V = T \thermfrac{S}{T}{V,N}, C_p = T \thermfrac{S}{T}{p,N}.
\]国内的教材常常将 $Q$ 前的字母 $d$ 加上横杠,成为表达式 $\bar{d}Q,$ 这是为了表达它不是全微分。它们对于功也采取了类似的写法。我们认为,这并不重要;但读者应该注意,在积分时只有对于全微分的积分才满足 Newton-Leibniz 公式,因为传热可能依赖于路径。做功不依赖于路径的条件是只有势力做功。

经过上述讨论,熵的表达式可以写为(为了方便,设压强是唯一的广义力,并设只有一种粒子):
\[
dS = \frac{dE}{T} + \frac{pdV}{T} -\frac{\mu dN}{T}.
\]


根据这个公式,我们可以讨论熵增原理导致的物体平衡条件。在讨论物体的平衡条件时,我们应当考虑一个小系统与环境大系统之间的总熵。按照我们之间所讨论的,我们知道熵应当取最大值。因此:
\[
S_0 = S_1 + S_{2}. (1\ll 2)
\]
最后一个式子是对于两个系统的体积,能量及粒子数说的。对此式取变分:
\[
\delta S_0 = \frac{\delta E_1}{T_1} + \frac{p_1\delta V_1}{T_1} -\frac{\mu_1 \delta N_1}{T_1} + \frac{\delta E_2}{T_2} + \frac{p_2\delta V_2}{T_2} -\frac{\mu_2 \delta N_2}{T_2}.
\]
考虑到能量守恒,体积守恒及粒子数守恒,我们有约束:
\[
\delta E_1 = -\delta E_2 = \delta E,\quad \delta V_1 = -\delta V_2 = \delta V,\quad \delta N_1 = -\delta N_2 = \delta N.
\]
于是有:
\[
\delta S_0 = \delta E\left( \frac1{T_1}-\frac1{T_2} \right) + \delta V\left( \frac{p_1}{T_1}-\frac{p_2}{T_2} \right) - \delta N\left( \frac{\mu_1}{T_1}-\frac{\mu_2}{T_2} \right) = 0.
\]
按照三个变量的独立性,我们就得到了平衡条件:
\begin{linenomath}\begin{equation}
\boxed{\begin{cases}
    T_1 = T_2 = T\\
    p_1 = p_2 = p.\\
    \mu_1 = \mu_2 = \mu
\end{cases}}
\label{equil_cond}
\end{equation}\end{linenomath}
进一步,我们可以对于熵取二阶变分,按照我们之前的讨论,这个二阶变分应该是负定的。在这里我们采取这样的方式:只看外部系统的熵变化,将它用内部参量表示。这就是说,让我们使用 $\delta S_1 = \delta S$ 作为一个独立的变分变量。
\[
\begin{aligned}
    \delta^2 S_0 &= \delta^2 S_2 \\ &= -\frac{\delta E\delta T}{T^2} + \frac{\delta p\delta V}{T} - \frac{\delta T\delta S}{T} - \frac{\delta\mu\delta N}{T} -\left( -p\delta V + T\delta S + \mu \delta N \right)\frac{\delta T}{T^2}\\
    &=  \frac{\delta p\delta V - \delta T\delta S - \delta\mu\delta N}{T}. \\
\end{aligned}
\]
最后一个等号使用了内能 $E$ 的展开。让我们使用 $(V,T,N)$ 作为独立的变分变量,则对于其他的变量我们有
\[
\begin{cases}
    \delta p &= \left( \dfrac{\partial p}{\partial V} \right)_{T,N} \delta V + \left(\dfrac{\partial p}{\partial T}\right)_{V,N} \delta T + \left( \dfrac{\partial p}{\partial N} \right)_{T,V} \delta N \\
    \delta S &= \left( \dfrac{\partial S}{\partial V} \right)_{T,N} \delta V + \left(\dfrac{\partial S}{\partial T}\right)_{V,N} \delta T + \left( \dfrac{\partial S}{\partial N} \right)_{T,V} \delta N. \\
    \delta \mu &= \left( \dfrac{\partial \mu}{\partial V} \right)_{T,N} \delta V + \left(\dfrac{\partial \mu}{\partial T}\right)_{V,N} \delta T + \left( \dfrac{\partial \mu}{\partial N} \right)_{T,V} \delta N
\end{cases}
\]
考虑到上节中得到的三个 Maxwell 关系,我们就有:
\[
\delta^2 S_0 = \frac1T\left( \left( \dfrac{\partial p}{\partial V} \right)_{T,N} \delta V^2 - \left(\dfrac{\partial S}{\partial T}\right)_{V,N} \delta T^2 - \left( \dfrac{\partial \mu}{\partial N} \right)_{T,V} \delta N^2 \right).
\]
因为熵的二阶变分是负定的,因此我们的稳定平衡条件就是:
\begin{linenomath}\begin{equation}
\begin{cases}
    \left( \dfrac{\partial p}{\partial V} \right)_{T,N} &< 0\\
    \left(\dfrac{\partial S}{\partial T}\right)_{V,N} &>0.\\
    \left( \dfrac{\partial \mu}{\partial N} \right)_{T,V} &>0
\end{cases}
\label{stab_cond}
\end{equation}\end{linenomath}
第二式就是说, $C_V>0.$

\section{其他热力学函数}
\subsection{内能的 Legendre 变换}
在物理中,Legendre 变换就是对于一个全微分式加上或减去一对或几对共轭变量的乘积的全微分。譬如对于全微分式 \[dZ(x,y) = Xdx + Ydy,\] 当我们减去 $d(Xx)$ 时,我们就得到新的量 \[d(Z-Xx) = -xdX + Ydy.\] 注意,这时候量 $Z' = Z -Xx$ 就变成 $X$ 与 $y$ 的函数。

内能有三对共轭变量,对于他们的 Legendre 变换(算上内能自己)一共有 8 种。但是考虑到实用性,我们除了内能 $E$ 外,最常用的有 4 种热力学函数:
\begin{itemize}
    \item 焓 $W := E+pV = W(p,S,N),$
    \item 自由能 $F := E-TS = F(V,T,N),$
    \item Gibbs 函数 $\Phi := E+pV-TS = W-TS = \Phi(p,T,N),$
    \item 热力学势 $\Omega := E-TS-\mu N = F-\mu N = \Phi(V,T,\mu).$
\end{itemize}
对于这些热力学函数,我们同样可以写出它们的全微分式:对于焓,
\begin{linenomath}\begin{equation}
dW = Vdp+TdS+\mu dN,
\end{equation}\end{linenomath}对于自由能,
\begin{linenomath}\begin{equation}
dF = -SdT-pdV+\mu dN,
\end{equation}\end{linenomath}对于吉布斯函数,
\begin{linenomath}\begin{equation}
d\Phi = -SdT + Vdp + \mu dN,
\end{equation}\end{linenomath}对于热力学势,
\begin{linenomath}\begin{equation}
d\Omega = -SdT - pdV - Nd \mu.
\end{equation}\end{linenomath}
\subsection{广延量与强度量\quad Gibbs-Duhem 关系}
回忆在 \ref{work} 节我们定义元功时,一般对于广义坐标要求是可加性的量。我们现在称这些坐标具有\textbf{广延量}的性质。注意,不是只有在表征空间中真的具有广延的才是广延量;所有热力学函数、粒子数和熵都是这样的量。与广延量共轭的量称为\textbf{强度量}。温度、压强、化学势和所有作用于宏观物体的力一般都是强度量。
既然如此,让我们将多粒子的情形下系统的热力学函数与只有一个粒子的系统热力学函数联系起来。这就是说:
\begin{linenomath}\begin{equation}
\frac{E(V,S,N)}{N} = E\left( \frac{V}{N},\frac{S}{N},1 \right),
\end{equation}\end{linenomath}
及
\begin{linenomath}\begin{equation}
\frac{F(V,T,N)}{N} = F\left( \frac{V}{N},T,1 \right),
\end{equation}\end{linenomath}
及
\begin{linenomath}\begin{equation}
\frac{\Phi(p,T,N)}{N} = \Phi\left( p,T,1 \right).
\end{equation}\end{linenomath}
但是另一方面,我们又有:
\begin{linenomath}\begin{equation}
\mu = \left( \frac{\partial E}{\partial N} \right)_{V,S} = \left( \frac{\partial F}{\partial N} \right)_{V,T} = \left( \frac{\partial \Phi}{\partial N} \right)_{p,T} = \Phi(p,T,1).
\end{equation}\end{linenomath}
这样就有
\begin{linenomath}\begin{equation}
\Phi(p,T,N) = N\mu(p,T).
\end{equation}\end{linenomath}
这就是说,\textbf{化学势就是单个粒子的 Gibbs 函数。} 因此,我们对于热力学势就有 $\Omega = F - \Phi = -pV.$ 因此直接写出全微分式就得到
\begin{linenomath}\begin{equation}
d\Omega  = -pdV - Vdp = -SdT - pdV - Nd \mu.
\end{equation}\end{linenomath}
因此得到 \textbf{Gibbs-Duhem 关系}:
\begin{linenomath}\begin{equation}
Vdp - SdT - Nd\mu = 0.
\end{equation}\end{linenomath}
这个关系是 $p$ 与 $T$ 作为强度量的必然结果。这意味着强度量之间不是独立的,它们之间至少存在一个约束。强度量的自由度个数在相变理论中将作详细讨论,最后将会得到称为 Gibbs 相律的结论。
\subsection{和统计结果的关联\quad 热力学函数无穷小增量的等价性}
现在设除了压强之外,系统与外界还通过一些力学参量连接起来. 那么现在就写有:
\begin{linenomath}\begin{equation}
dE= pdV + TdS + \mu dN -\sum_{i} Y_idy_i,
\end{equation}\end{linenomath}
及\begin{linenomath}\begin{equation}
dW = Vdp+TdS+\mu dN-\sum_{i} Y_idy_i,
\end{equation}\end{linenomath}及
\begin{linenomath}\begin{equation}
dF = -SdT-pdV+\mu dN-\sum_{i} Y_idy_i,
\end{equation}\end{linenomath}及
\begin{linenomath}\begin{equation}
d\Phi = -SdT + Vdp + \mu dN-\sum_{i} Y_idy_i,
\end{equation}\end{linenomath}及
\begin{linenomath}\begin{equation}
d\Omega = -SdT - pdV - Nd \mu-\sum_{i} Y_idy_i.
\end{equation}\end{linenomath}
我们在 \ref{work} 节中,已经藉由内能的全微分式推出了将微观与宏观联系在一起的 (\ref{macro_micro1}) 式。现在我们首先注意到下面的式子:
\begin{linenomath}\begin{equation}
\left( dE \right)_{V,S,N}=\left( dW \right)_{p,S,N} = \left( dF \right)_{T,V,N} = \left(d\Phi\right)_{T,p,N} = \left(d\Omega\right)_{T,V,\mu}.
\end{equation}\end{linenomath}
这个式子被 Landau 称为\textbf{小增量定理,}这表示在固定对应参量的情况下,对每一个热力学函数取微元都得到同样的结果。所以我们直接将(\ref{macro_micro1}) 式中的 $E$ 换为 $F$ 及 $\Omega,$ 就可以得到:
\begin{linenomath}\begin{equation}
\frac{\partial \bar E}{\partial y_i} = \left( \frac{\partial  F}{\partial y_i} \right)_{T,V,N,\text{其他}y_j}\label{macro_micro2}
\end{equation}\end{linenomath}
及
\begin{linenomath}\begin{equation}
\frac{\partial \bar E}{\partial y_i} = \left( \frac{\partial  \Omega}{\partial y_i} \right)_{T,V,\mu,\text{其他}y_j}\label{macro_micro3}
\end{equation}\end{linenomath}
我们看到,这就是正则系综与巨正则系综所固定的参量,因此自然可以想见它们与对应的分布函数的密切联系。我们进一步注意到自由能与热力学势具有这样的普适性质,就是\textbf{内能可以用它们的偏导表示}。具体来说,我们对于自由能:
\[
S = -\thermfrac{F}{T}{p,N,y_i},\quad E = F + TS = \thermfrac{\left(\frac FT  \right)}{\frac{1}{T}}{p,N,y_i}.
\]
我们将这个式子与在推导平均能量时得到的 \[E = -\ppp{\ln Z}{\beta}\quad \left( \beta = \frac{1}{kT} \right)\] 相比较,我们就得到了精确等式:
\begin{linenomath}\begin{equation}
F = -kT\ln Z.
\end{equation}\end{linenomath}
这样就可以写出正则分布的准确形式:\begin{linenomath}\begin{equation}
\rho_{\text{canon}} = \exp\left( \frac{F-\hat{H}}{kT} \right).
\end{equation}\end{linenomath}
 类似的,对于热力学势,有:
\begin{linenomath}\begin{equation}
\Omega = -kT\ln \Xi.
\end{equation}\end{linenomath}
同样可以写出
\begin{linenomath}\begin{equation}
\rho_{\text{grand}} = \exp\left( \frac{\Omega - \hat{H} + \mu \hat{N}}{kT} \right).
\end{equation}\end{linenomath}


\subsection{其他 Maxwell 关系}
按照 Maxwell 关系导出的一般法则,我们还可以写出下列 Maxwell 关系式:(这里只限于写出粒子数不变的关系)
\begin{linenomath}\begin{equation}
\left( \frac{\partial T}{\partial p} \right)_S = \left( \frac{\partial V}{\partial S} \right)_p
\label{maxwell2}
\end{equation}\end{linenomath}
及
\begin{linenomath}\begin{equation}
\left( \frac{\partial S}{\partial V} \right)_T = \left( \frac{\partial p}{\partial T} \right)_V
\label{maxwell3}
\end{equation}\end{linenomath}
及
\begin{linenomath}\begin{equation}
\left( \frac{\partial S}{\partial p} \right)_T = -\left( \frac{\partial V}{\partial T} \right)_p.
\label{maxwell4}
\end{equation}\end{linenomath}
化学势不变的 Maxwell 关系可以由热力学势以及它的 Legendre 变换得出。

\section{热力学量间的关系}
\subsection{Jacobi 行列式}
我们先介绍热力学关系的导出中的一个重要工具,就是 \textbf{Jacobi 行列式}技术。 Jacobi 行列式是下列行列式:
\[
\jacobian{u}{v}{x}{y} := \left| \begin{matrix}
    \ppp{u}{x} & \ppp{u}{y} \\
    \ppp{v}{x} & \ppp{v}{y}
\end{matrix} \right|.
\]
熟悉的读者可能认出,这就是坐标变换的行列式。在热力学中,我们主要关心这个行列式的下述性质:
\begin{itemize}
    \item 反对称性:\[\jacobian{v}{u}{x}{y} = - \jacobian{u}{v}{x}{y}.\]
    \item 保持一个参量的求导:\[\jacobian{u}{y}{x}{y} = \ppp{u}{x} - \ppp{u}{y}\ppp{y}{x} =  \thermfrac{u}{x}{y}.\]
    \item 链式法则:\[\jacobian{u}{v}{x}{y} = \jacobian{u}{v}{a}{b}\cdot\jacobian{a}{b}{x}{y}.\]
\end{itemize}

现在来看比 $\jacobian{p}{V}{T}{S}.$ 由上面的性质,我们有:
\[
    \jacobian{p}{V}{T}{S} = \jacobian{p}{V}{T}{V}\jacobian{T}{V}{T}{S} = \frac{\thermfrac{p}{T}{V}}{\thermfrac{S}{V}{T}} = \frac{\thermfrac{p}{T}{V}}{\thermfrac{p}{T}{V}} = 1.
\]
其中已经用到 Maxwell 关系 (\ref{maxwell3}). 现在我们继续研究热力学量之间的关系。
\subsection{热容与热容的导数}
我们首先研究热容如何表示为热力学函数的导数。我们有:\[
C_V = T\thermfrac{S}{T}{V,N} = \thermfrac{E}{T}{V,N}
\]及\[
C_p = T\thermfrac{S}{T}{p,N} = \thermfrac{W}{T}{p,N}.
\]
又因为\[
\thermfrac{E}{V}{T} = T\thermfrac{S}{V}{T}-p=T\thermfrac{P}{T}{V}-p
\]以及
\[
\thermfrac{W}{P}{T} = -T\thermfrac{V}{T}{p}+V,
\]
我们现在就可以写出全微分式:
\[
dE = C_VdT + \left( T\thermfrac{p}{T}{V}-p \right)dV,
\]及
\[
dW = C_pdT + \left( -T\thermfrac{V}{T}{p}+V \right)dp.
\]
因此按照 Cauchy-Riemann 条件,我们有:
\[
\thermfrac{C_V}{V}{T} = T \left( \frac{\partial^2 p }{\partial T^2} \right)_V
\]
及
\[
\thermfrac{C_p}{p}{T} = -T \left( \frac{\partial^2 V }{\partial T^2} \right)_p.
\]
如此,我们就可以通过系统的状态方程来计算系统的热容。接下来,关于 $C_p$ 和 $C_V$ 的关系我们有
\[
\begin{aligned}
    C_v &= T\thermfrac{S}{T}{V} = T \jacobian{S}{V}{T}{V} = T\frac{\jacobian{S}{V}{T}{p}}{\jacobian{T}{V}{T}{p}}\\
    &= T\frac{\thermfrac{S}{T}{p}\thermfrac{V}{p}{T} - \thermfrac{S}{p}{T}\thermfrac{V}{T}{p}}{\thermfrac{V}{p}{T}} \\
    &= C_p-T\frac{\thermfrac{S}{p}{T}\thermfrac{V}{T}{p}}{\thermfrac{V}{p}{T}} = C_p-T\frac{\thermfrac{V}{T}{p}^2}{\thermfrac{V}{p}{T}}.
\end{aligned}
\]
对于稳定的系统,正如我们在 (\ref{stab_cond}) 的第一式中所指出的那样,导数 $\thermfrac{V}{P}{T}$ 总是负的,因此我们推出:
\begin{linenomath}\begin{equation}
C_p>C_V.
\end{equation}\end{linenomath}
用 $p$ 的相关公式,我们也可以得到:
\[
C_p - C_V = -T\frac{\thermfrac{p}{T}{V}^2}{\thermfrac{p}{V}{T}}.
\]

\subsection{绝热条件下状态参量的导数}
下面的公式可以适用于宏观物体准静态绝热地膨胀(或压缩)时。在这类过程中,熵 $S$ 保持不变。对于绝热膨胀的温度变化:
\[
\thermfrac{T}{V}{S} = \jacobian{T}{S}{V}{S} = \jacobian{T}{S}{V}{T} / \jacobian{V}{S}{V}{T} = -\frac{\thermfrac{S}{V}{T}}{\thermfrac{S}{T}{V}} = -\frac{T}{C_V}\thermfrac{p}{T}{V}.
\]
对于绝热加压的温度变化:
\[
\thermfrac{T}{p}{S} = \frac{T}{C_p} \thermfrac{V}{T}{p}.
\]
而对于绝热过程中的压强与体积也可以类似写出:
\[
\thermfrac{V}{P}{S} = \jacobian{V}{S}{P}{S} = \frac{\jacobian{V}{S}{V}{T}\jacobian{V}{T}{P}{T}}{\jacobian{P}{S}{P}{T}} = \frac{C_V}{C_p} \thermfrac{V}{P}{T}.
\]
利用上节推出的关于 $C_p-C_V$ 的公式,我们可以写出关于 $C_p$ 的公式:
\[
\thermfrac{V}{p}{S} = \thermfrac{V}{p}{T} + \frac{T}{C_p}\thermfrac{V}{T}{p}
\] 及关于 $C_V$ 的公式 \[
\thermfrac{p}{V}{S} = \thermfrac{p}{V}{T} - \frac{T}{C_V}\thermfrac{p}{T}{V}.
\]
\section{现实热力学问题}
下面讨论几个现实中首先会碰到的热力学问题。
\subsection{温标的测定}
对于一个任意的温度计,它标示的是某个经验温标 $\tau.$ 然而,我们想要知道它与绝对温标 $T$ 的关系,为了方便,可以假设两者是一一对应的。我们的思路是利用 Maxwell 关系:对于传热 $dQ,$ 可以写出
\[
\thermfrac{Q}{p}{T} = -T\thermfrac{S}{p}{T} = -T\thermfrac{V}{T}{p}.
\]
现在利用一一对应的性质,我们有
\[
\thermfrac{Q}{p}{T} = -T\thermfrac{V}{\tau}{p} \frac{d\tau}{dT}.
\]
这样 \[
\frac{d\ln T}{d\tau} = -\frac{\thermfrac{V}{\tau}{p}}{\thermfrac{Q}{p}{\tau}}.
\]
右边的两项可以设计等压与等温过程,分别测定(后者需要从外界间接测量)。
\subsection{气体的 Joule-Thomson 过程}
现在

\subsection{各种条件下的最大做功}

\subsection{热机\quad Carnot 循环}



\chapter{过渡到准经典的情形}
\section{准经典波函数与能量}
\subsection{WKB 波函数}
\section{准经典 Liouville 定理}
\section{分布函数与简并度}
\section{全同性的准经典对应}
\part{气体的一般理论}


\part{凝聚体的统计理论}

\part{相变理论}

\part{气体和凝聚体的统计场论}

\appendix

\part{附录}
\chapter{量子多体问题和二次量子化}
量子统计力学所研究的问题是\textbf{大量}服从 Schrodinger 方程的粒子的 \textbf{平衡态统计行为}。对于经典粒子来说,在粒子数变多时,解出这些粒子所服从的 Hamilton 方程已经不可能,找出满足它们的初值条件的特解更是不可能。在量子的情形下,要考虑这些粒子的相互作用下再解出 Schrodinger 方程,为人来说已经不可能。但是,粒子的大量性可以使我们做出一些遍历性的假设,进而发现这些粒子所满足的统计规律,这些规律不依赖于所要求的 Schrodinger 方程的初值条件和通解。

\section{量子多体问题的表述}
一个量子多体问题本质上是一个 Schrodinger 方程的求解问题:
\begin{linenomath}\begin{equation}
\frac{\partial}{\partial t} \Psi  = \hat H\left(\hat{\vec{q}}_1,\hat{\vec{p}}_1,\cdots,\hat{\vec{q}}_n,\hat{\vec{p}}_n\right) \Psi =  \sum_{i = 1}^{n} \left( \frac{\hat{\vec{p}}_i^2 }{2m}  + U (\hat{\vec{q}}_i ,t) \right)\Psi + V_{int} \left (\hat{\vec{q}}_1,\cdots,\hat{\vec{q}}_n \right )\Psi.
\label{schrodinger}\end{equation}\end{linenomath}

在量子统计物理中,我们大多研究的是全同粒子的统计问题。因此,在 (\ref{schrodinger}) 中首先有 \[m_i = m \quad (i = 1,2,\cdots, n).\]
其次,粒子间的相互作用项形如:
\[
\hat V_{int} = \frac{1}{2!}\sum_{i\neq j} \hat{V_2}(\hat{\vec q}_i,\hat{\vec q}_j,t) + \frac{1}{3!}\sum_{ i \neq j \neq k} \hat{V_3}(\hat {\vec q}_i, \hat {\vec q}_j, \hat {\vec q}_k,t) + \cdots
\]

粒子的全同性首先表现在 Hamilton 量上:当对于上面写出的哈密顿量 $H$ 实行变换 \[\left( \hat{\vec {q}}_i,\hat{\vec p}_i \right) \leftrightarrow \left(\hat{\vec q}_j,\hat{\vec p}_j\right)\] 时,哈密顿量的形式不变。这要求对于每一个相互作用势能项对于坐标 $\{\hat{\vec{q}}_i\}$ 的置换保持不变——这是符合直觉的。于是,我们由于 Hamilton 量具有这样的对称性,自然想到波函数具有这样的对称性:也就是对于坐标与动量的对换 $\left( \hat{\vec {q}}_i,\hat{\vec p}_i \right) \leftrightarrow \left(\hat{\vec q}_j,\hat{\vec p}_j\right),$ 在坐标表象下成立
\begin{linenomath}\begin{equation}
\Psi\left(\vec{q}_1,\cdots,\vec{q}_i,\cdots,\vec{q}_j,\cdots,\vec{q}_n\right) = e^{i\alpha}\Psi\left(\vec{q}_1,\cdots,\vec{q}_j,\cdots,\vec{q}_i,\cdots,\vec{q}_n\right). 
\label{commuta}
\end{equation}\end{linenomath}
其中考虑到了波函数模方的不变性。 这个式子的物理意义是:\textbf{粒子的交换不带来物理态的改变,或全同粒子的分辨在物理上是不可能的。} 两种自然的情况是 $\alpha = 0$ 和 $\alpha = \pi,$ 在这些情况下粒子分别被称为满足 \textbf{Bose 统计} 与 \textbf{Fermi 统计}。

\section{二次量子化和 Fock 空间}
我们只考虑一个粒子的情形,这时 Hamilton 量是:
\[
\hat H = \frac{\hat{\vec{p}}^2 }{2m}  + U (\hat{\vec{q}} ,t)
\]
我们由 Schrodinger 方程可以解出一组完备的解 $\left\{\psi_s(\vec{q})\right\},$ 其中 $s$ 是某些量子数。这些波函数张成 \textbf{波函数 Hilbert 空间} $L^2(\mathbb{R} )$ 的一组基底,因此任何函数可用这一组本征函数展开。

于是我们现在可以构造多粒子波函数的空间,它是$n$个单粒子波函数空间的 \textbf{直积空间。}  直觉上它的基底是形如 
\[
\psi_{s_1}(\vec{q}_1) \otimes \psi_{s_2}(\vec{q}_2) \otimes \cdots \otimes \psi_{s_n}(\vec{q}_n)
\]
的向量。然而,我们发现这样的选择不是物理的:因为它们并不能反映出粒子的全同性。事实上,服从 Fermi 统计的波函数与服从 Bose 统计的波函数构成这个直积空间的两个无穷维子空间(这可以由 (\ref{commuta}) 得到,只需要注意到(反)对称波函数的线性组合还是(反)对称波函数),这两个子空间被称为 \textbf{Fock 空间}。 张成它们的基底才是我们关心的对象。
\subsection{基向量的构造}
我们现在实行一种 \textbf{对称化} 与 \textbf{反对称化} 操作,这样它们便可以张成对应的子空间。利用置换,我们可以写:
\begin{linenomath}\begin{equation}
\Psi_B(s_1,s_2,\cdots,s_n)\left[\vec{q}_1,\cdots,\vec{q}_n\right] = N \sum_{\pi\in S_n}\bigotimes_{i = 1}^{n} \psi_{s_{\pi(i)}}\left(\vec{q}_i\right)
\end{equation}\end{linenomath}
以及
\begin{linenomath}\begin{equation}
\Psi_F(s_1,s_2,\cdots,s_n)\left[\vec{q}_1,\cdots,\vec{q}_n\right] = N' \sum_{\pi\in S_n}(-1)^{\pi}\bigotimes_{i = 1}^{n} \psi_{s_{\pi(i)}}\left(\vec{q}_i\right)
\end{equation}\end{linenomath}
其中 $(-1)^{\pi}$ 是置换 $\pi$ 的符号,$N$ 和 $N'$ 是归一化常数,他们的值将在稍后确定。但是,显而易见的是现在的波函数的确是(反)对称的。他们的正交性可以验证,只需要注意到一次量子化的波函数具有的正交归一性。

对于 Bose 统计,我们有:
\[
    \begin{aligned}
        \Psi_B(s_1,s_2,s_3,\cdots,s_n)\Psi_B(s_1',s_2',s_3',\cdots,s_n') &= NN'\sum_{\pi}\sum_{\pi'}\int \prod_{i=1}^{n} d^3\vec{q}_i \psi^*_{s_{\pi(i)}}(\vec{q}_i)\psi^{}_{s_{\pi'(i')}}(\vec{q}_i) \\
        &= NN' \sum_{\pi}\sum_{\pi'} \delta_{s_{\pi(i)}s_{\pi'(i')}} \\
        &= n!NN' \sum_{\pi}\delta_{s_{\pi(i)i'}}.
    \end{aligned}
\]
对于处于不同物理态的粒子,我们看到上面的内积等于$0$。对于同一个态的内积,我们注意到这些$s_i$ 可能是重复的,所以做计算时应该注意。

\[
N = \frac{1}{\sqrt{n!\prod_s n_s!}},
\]
$n_s$ 是$s$所标示的能级的占有数。

对于 Fermi 统计,我们同样可以做计算:
\[
    \begin{aligned}
        \Psi_F(s_1,s_2,s_3,\cdots,s_n)\Psi_F(s_1',s_2',s_3',\cdots,s_n') &= NN'\sum_{\pi}\sum_{\pi'}\int \prod_{i=1}^{n} d^3\vec{q}_i \psi^*_{s_{\pi(i)}}(\vec{q}_i)\psi^{}_{s_{\pi'(i')}}(\vec{q}_i) (-1)^\pi(-1)^{\pi'}\\
        &= NN' \sum_{\pi}\sum_{\pi'} \delta_{s_{\pi(i)}s_{\pi'(i')}} (-1)^\pi(-1)^{\pi'}\\
        &= n!NN' \sum_{\pi}\delta_{s_{\pi(i)i'}}(-1)^{\pi}.
    \end{aligned}
\]

同样,对于处于不同物理态的粒子,我们看到上面的内积等于$0$。对于 Fermi 统计还需注意,每一个能级上至多只能占有一个粒子(如果不然,那么交换两个处于同一能级上的粒子将得到波函数为0)。因此:
\[
N = \frac{1}{\sqrt{n!}},
\]
因为 $n_s = 0 $ 或 $1.$

我们看到,这个表象下的基底自然变成了\textbf{占有数表象,} 也就是说,我们可以将一个态写成一个列表 $\ket{[n_s]},$ $n_s$ 是能级 $s$ 的占有数。对于 Fermi 统计,每个能级的占有数只能为 $0$ 或 $1;$ 对于 Bose 统计则没有此限。

\subsection{创造算符与湮灭算符}
我们对于量子谐振子,我们构造了两个算符:\textbf{创造算符} 和 \textbf{湮灭算符}(统称为\textbf{升降算符}),他们将粒子数低的态映射至粒子数高的态。对于角动量理论,我们同样可以构造这样的算符。现在可以看到,在占有数表象下升降算符的引入不仅是必要的,而且是自然的。严格来说,创造算符应该定义为映射:
\[
b^\dagger: F_n \to F_{n+1},
\]
而湮灭算符则应当为定义为映射:
\[
b: F_n \to F_{n-1},
\]
而整个 Fock 空间则是这些 $n-$ 粒子子空间的\textbf{无交直积}。
对于 Fermi 统计和 Bose 统计,升降算符的表现并不相同。

对于 $n-$ 粒子态来说,我们定义算符 $b_s^\dagger$ 和 $b_s,$ 就像量子力学中那样,他们分别对应创造算符和湮灭算符。
\[
    b_{s_i}^\dagger \Psi = \frac{1}{\sqrt{n+1}}\psi_{s_i} \otimes_\pm \Psi,
\]
\[
    b_{s_i}\Psi = \frac{1}{\sqrt{n}}\psi_{s_i} \oslash_\pm \Psi,
\]
其中,$\otimes_+$ 代表对称地插入,$\oslash_+$ 代表对称地删除;换为减号的各个算符代表反对称的操作(对于第 $i$ 个位置乘因子 $(-1)^{i-1}$)。

譬如,对于服从 Bose 统计的双态全同粒子来说,可以用态 $\ket{m,n}$ 来标记态,其中 $m, n$ 是非负整数,分别象征低能级和高能级的占有数。
现在对于态 $\ket{1,1}=\dfrac{1}{\sqrt 2} (\psi_1\psi_2+\psi_2\psi_1),$ 看对于低能态的创造算符与湮灭算符:
\[
\begin{aligned}
    b_1^\dagger\ket{1,1} &= \frac{1}{\sqrt{2}}\left(b_1^\dagger\psi_1\psi_2 + b_1^\dagger\psi_2\psi_1\right)\\
    &= \frac{1}{\sqrt{2}}\left( \dfrac{2}{\sqrt{3}}\left( \psi_1 \psi_1 \psi_2 + \psi_1 \psi_2 \psi_1 + \psi_2 \psi_1 \psi_1\right) \right)\\
    &=\sqrt{2} \ket{2,1},
\end{aligned}
\]
以及类似有
\[
    b_1^\dagger\ket{1,1} =  \ket{0,1}.
\]
一般的,对于服从 Bose 统计的粒子或 Bose 子,我们有 
\begin{linenomath}\begin{equation}
{b}_{\alpha }^{ \dagger  }|\cdots ,{n}_{\beta },{n}_{\alpha },{n}_{\gamma },\cdots \rangle  = \sqrt{{n}_{\alpha } + 1}|\cdots ,{n}_{\beta },{n}_{\alpha } + 1,{n}_{\gamma },\cdots \rangle , 
\end{equation}\end{linenomath}
 
\begin{linenomath}\begin{equation}
{b}_{\alpha }|\cdots ,{n}_{\beta },{n}_{\alpha },{n}_{\gamma },\cdots \rangle  = \sqrt{{n}_{\alpha }}|\cdots ,{n}_{\beta },{n}_{\alpha } - 1,{n}_{\gamma },\cdots \rangle . 
\end{equation}\end{linenomath}

而且若指定了一个真空态$\ket{0},$ 我们便可写出一般离子态的生成公式:
\begin{linenomath}\begin{equation}
\ket{[n_s]} = \prod_s \frac{(\hat{b}_s^\dagger)^{n_s}}{\sqrt{n_s!}}\ket{0}.
\end{equation}\end{linenomath}


而对于 Fermi 子来说也有类似的公式,只需要注意到 Fermi 子满足态的不相容性,以及插入的反对称系数和可能带来的负号。
例如,对于服从 Fermi 统计的双态全同粒子来说,可以用态 $\ket{m,n}$ 来标记态,其中 $m, n$ 是 $0$ 或 $1.$ 让我们同样看态 $\ket{1,1}_- =\dfrac{1}{\sqrt 2} (\psi_1\psi_2-\psi_2\psi_1),$ 这里的 $-$ 标明的是态的反对称性。经过计算我们有

\[
    c_1^\dagger \ket{1,1} = 0,
\]
以及
\[
    c_1 \ket{1,1} = -\ket{1,0}.
\]

对于 Fermi 子,我们同样可以写出一般公式: 
\begin{linenomath}\begin{equation}
{c}_{\alpha }^{ \dagger  }|\cdots ,{n}_{\beta },{n}_{\alpha },{n}_{\gamma },\cdots \rangle  = ( - 1{)}^{\mathop{\sum }\limits_{{\beta  < \alpha }}{n}_{\beta }}\sqrt{1 - {n}_{\alpha }}|\cdots ,{n}_{\beta },1 + {n}_{\alpha },{n}_{\gamma },\cdots \rangle 
\end{equation}\end{linenomath}
与
\begin{linenomath}\begin{equation}
{c}_{\alpha }|\cdots ,{n}_{\beta },{n}_{\alpha },{n}_{\gamma },\cdots \rangle  = ( - 1{)}^{\mathop{\sum }\limits_{{\beta  < \alpha }}{n}_{\beta }}\sqrt{{n}_{\alpha }}|\cdots ,{n}_{\beta },1 - {n}_{\alpha },{n}_{\gamma },\cdots \rangle . \end{equation}\end{linenomath}
但显然这个公式的系数是配凑得到的,不是自然的。
\subsection{正则对易关系与场算符}
对于 Bose 子来说,我们对于升降算符,正如同量子力学中那样,可以验证对易关系
\begin{linenomath}\begin{equation}
\left[ b_s,b_{s'} \right] = \left[ b_s^\dagger, b_{s'}^\dagger \right] = 0, \left[ b_s,b_{s'}^\dagger \right] = \delta_{ss'}.
\end{equation}\end{linenomath}
对于 Fermi 子来说,因为反对称性的要求,我们要将对易关系换成反对易关系:
\begin{linenomath}\begin{equation}
\left[ c_s,c_{s'} \right]_- = \left[ c_s^\dagger, c_{s'}^\dagger \right]_- = 0, \left[ c_s,c_{s'}^\dagger \right]_- = \delta_{ss'}.
\end{equation}\end{linenomath}

对于一个配备了升降算符的 Fock 空间( Fermi 子或 Bose 子),我们便可以定义场算符
\[
\hat\Psi \left( \vec{q}\right)  = \mathop{\sum }\limits_{s }{\psi }_{s}\left( \vec{q}\right) {a}_{s } ,
\]
及
\[
{\hat\Psi }^{ \dagger  }\left( \vec{q}\right)  = \mathop{\sum }\limits_{s}{\psi }_{s}^{ * }\left( \vec{q}\right) {a}_{s }^{ \dagger  }.
\]
粗略的来说,这些场算符(和它的 Hermite 共轭)是在$\vec{q}$处湮灭(对应地,产生)粒子的算符。这些算符具有对易关系
\[
\left[ \hat\Psi(\vec{q}),\hat\Psi^\dagger(\vec{\tilde q}) \right]_{\pm} = \delta^3(\vec{q}-\vec{\tilde q}).
\]
这就是所谓二次量子化的波函数。 我们看到,二次量子化的一般方法就是写出这样的场算符形式。在量子场论当中这种形式的出现是自然的。

作为示例,我们考虑 $U=0$ 的情形,这时单粒子波函数(如用箱归一化)是形如
\[
\psi_{\vec{p}}(\vec{q}) = \frac{1}{\sqrt{V}} \exp{\left( \frac{i}{\hbar}\vec{p}\cdot\vec{q} \right)}
\]
的波函数,因此场算符具有形式:
\[
\hat\Psi(\vec{q}) = \frac{1}{\sqrt{V}}\sum_{\vec{p}}\exp{\left( \frac{i}{\hbar}\vec{p}\cdot\vec{q} \right)} a_{\vec{p}}.
\]

\end{document}

