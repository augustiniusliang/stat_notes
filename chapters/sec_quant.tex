\chapter{量子多体问题和二次量子化}
量子统计力学所研究的问题是\textbf{大量}服从 Schrodinger 方程的粒子的 \textbf{平衡态统计行为}。对于经典粒子来说,在粒子数变多时,解出这些粒子所服从的 Hamilton 方程已经不可能,找出满足它们的初值条件的特解更是不可能。在量子的情形下,要考虑这些粒子的相互作用下再解出 Schrodinger 方程,为人来说已经不可能。但是,粒子的大量性可以使我们做出一些遍历性的假设,进而发现这些粒子所满足的统计规律。

\section{量子多体问题的表述}
一个量子多体问题本质上是一个 Schrodinger 方程的求解问题:
\begin{linenomath}\begin{equation}
\frac{\partial}{\partial t} \Psi  = \hat H\left(\hat{\vec{q}}_1,\hat{\vec{p}}_1,\cdots,\hat{\vec{q}}_n,\hat{\vec{p}}_n\right) \Psi =  \sum_{i = 1}^{n} \left( \frac{\hat{\vec{p}}_i^2 }{2m}  + U (\hat{\vec{q}}_i ,t) \right)\Psi + V_{int} \left (\hat{\vec{q}}_1,\cdots,\hat{\vec{q}}_n \right )\Psi.
\label{schrodinger}\end{equation}\end{linenomath}

在量子统计物理中,我们大多研究的是全同粒子的统计问题。因此,在 (\ref{schrodinger}) 中首先有 \[m_i = m \quad (i = 1,2,\cdots, n).\]
其次,粒子间的相互作用项形如:
\[
\hat V_{int} = \frac{1}{2!}\sum_{i\neq j} \hat{V_2}(\hat{\vec q}_i,\hat{\vec q}_j,t) + \frac{1}{3!}\sum_{ i \neq j \neq k} \hat{V_3}(\hat {\vec q}_i, \hat {\vec q}_j, \hat {\vec q}_k,t) + \cdots
\]

粒子的全同性首先表现在 Hamilton 量上:当对于上面写出的哈密顿量 $H$ 实行变换 \[\left( \hat{\vec {q}}_i,\hat{\vec p}_i \right) \leftrightarrow \left(\hat{\vec q}_j,\hat{\vec p}_j\right)\] 时,哈密顿量的形式不变。这要求对于每一个相互作用势能项对于坐标 $\{\hat{\vec{q}}_i\}$ 的置换保持不变——这是符合直觉的。于是,我们由于 Hamilton 量具有这样的对称性,自然想到波函数具有这样的对称性:也就是对于坐标与动量的对换 $\left( \hat{\vec {q}}_i,\hat{\vec p}_i \right) \leftrightarrow \left(\hat{\vec q}_j,\hat{\vec p}_j\right),$ 在坐标表象下成立
\[
\Psi\left(\vec{q}_1,\cdots,\vec{q}_i,\cdots,\vec{q}_j,\cdots,\vec{q}_n\right) = e^{i\alpha}\Psi\left(\vec{q}_1,\cdots,\vec{q}_j,\cdots,\vec{q}_i,\cdots,\vec{q}_n\right).
\]
其中考虑到了波函数模方的不变性。 这个式子的物理意义是:\textbf{粒子的交换不带来物理态的改变,或粒子的分辨在物理上是不可能的。} 两种自然的情况是 $\alpha = 0$ 和 $\alpha = \pi,$ 在这些情况下粒子分别被称为满足 \textbf{Bose 统计} 与 \textbf{Fermi 统计}。

我们希望写出一个便捷的 Hamilton 量与波函数的表示形式,尤其是计及粒子的全同性,为方便量子统计的计算。

\section{多粒子波函数和 Fock 空间}
我们只考虑一个粒子的情形,这时 Hamilton 量是:
\[
\hat H = \frac{\hat{\vec{p}}^2 }{2m}  + U (\hat{\vec{q}} ,t)
\]
我们由 Schrodinger 方程可以解出一组完备的解 $\left\{\psi_s(\vec{q})\right\},$ 其中 $s$ 是某些量子数。


