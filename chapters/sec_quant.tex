\chapter{量子多体问题和二次量子化}
量子统计力学所研究的问题是\textbf{大量}服从 Schrodinger 方程的粒子的 \textbf{平衡态统计行为}。对于经典粒子来说,在粒子数变多时,解出这些粒子所服从的 Hamilton 方程已经不可能,找出满足它们的初值条件的特解更是不可能。在量子的情形下,要考虑这些粒子的相互作用下再解出 Schrodinger 方程,为人来说已经不可能。但是,粒子的大量性可以使我们做出一些遍历性的假设,进而发现这些粒子所满足的统计规律。

\section{量子多体问题的表述}
一个量子多体问题本质上是一个 Schrodinger 方程的求解问题:
\begin{linenomath}\begin{equation}
\frac{\partial}{\partial t} \Psi  = \hat H\left(\hat{\vec{q}}_1,\hat{\vec{p}}_1,\cdots,\hat{\vec{q}}_n,\hat{\vec{p}}_n\right) \Psi =  \sum_{i = 1}^{n} \left( \frac{\hat{\vec{p}}_i^2 }{2m}  + U (\hat{\vec{q}}_i ,t) \right)\Psi + V_{int} \left (\hat{\vec{q}}_1,\cdots,\hat{\vec{q}}_n \right )\Psi.
\label{schrodinger}\end{equation}\end{linenomath}

在量子统计物理中,我们大多研究的是全同粒子的统计问题。因此,在 (\ref{schrodinger}) 中首先有 \[m_i = m \quad (i = 1,2,\cdots, n).\]
其次,粒子间的相互作用项形如:
\[
\hat V_{int} = \frac{1}{2!}\sum_{i\neq j} \hat{V_2}(\hat{\vec q}_i,\hat{\vec q}_j,t) + \frac{1}{3!}\sum_{ i \neq j \neq k} \hat{V_3}(\hat {\vec q}_i, \hat {\vec q}_j, \hat {\vec q}_k,t) + \cdots
\]

粒子的全同性首先表现在 Hamilton 量上:当对于上面写出的哈密顿量 $H$ 实行变换 \[\left( \hat{\vec {q}}_i,\hat{\vec p}_i \right) \leftrightarrow \left(\hat{\vec q}_j,\hat{\vec p}_j\right)\] 时,哈密顿量的形式不变。这要求对于每一个相互作用势能项对于坐标 $\{\hat{\vec{q}}_i\}$ 的置换保持不变——这是符合直觉的。于是,我们由于 Hamilton 量具有这样的对称性,自然想到波函数具有这样的对称性:也就是对于坐标与动量的对换 $\left( \hat{\vec {q}}_i,\hat{\vec p}_i \right) \leftrightarrow \left(\hat{\vec q}_j,\hat{\vec p}_j\right),$ 在坐标表象下成立
\begin{linenomath}\begin{equation}
\Psi\left(\vec{q}_1,\cdots,\vec{q}_i,\cdots,\vec{q}_j,\cdots,\vec{q}_n\right) = e^{i\alpha}\Psi\left(\vec{q}_1,\cdots,\vec{q}_j,\cdots,\vec{q}_i,\cdots,\vec{q}_n\right). 
\label{commuta}
\end{equation}\end{linenomath}
其中考虑到了波函数模方的不变性。 这个式子的物理意义是:\textbf{粒子的交换不带来物理态的改变,或粒子的分辨在物理上是不可能的。} 两种自然的情况是 $\alpha = 0$ 和 $\alpha = \pi,$ 在这些情况下粒子分别被称为满足 \textbf{Bose 统计} 与 \textbf{Fermi 统计}。


\section{二次量子化和 Fock 空间}
我们只考虑一个粒子的情形,这时 Hamilton 量是:
\[
\hat H = \frac{\hat{\vec{p}}^2 }{2m}  + U (\hat{\vec{q}} ,t)
\]
我们由 Schrodinger 方程可以解出一组完备的解 $\left\{\psi_s(\vec{q})\right\},$ 其中 $s$ 是某些量子数。这些波函数张成 \textbf{波函数 Hilbert 空间} $L^2(\mathbb{R} )$ 的一组基底,因此任何函数可用这一组本征函数展开。

于是我们现在可以构造多粒子波函数的空间,它是$n$个粒子的 \textbf{直积空间。}  直觉上它的基底是形如 
\[
\psi_{s_1}(\vec{q}_1) \otimes \psi_{s_2}(\vec{q}_2) \otimes \cdots \otimes \psi_{s_n}(\vec{q}_n)
\]
的向量。然而,我们发现这样的选择不是物理的:因为它们并不能反映出粒子的全同性。事实上,服从 Fermi 统计的波函数与服从 Bose 统计的波函数构成这个直积空间的两个无穷维子空间(这可以由 (\ref{commuta}) 得到,只需要注意到(反)对称波函数的线性组合还是(反)对称波函数),张成它们的基底才是我们关心的对象。

我们现在实行一种 \textbf{对称化} 与 \textbf{反对称化} 操作,这样它们便可以张成对应的子空间。利用置换,我们可以写:
\begin{linenomath}\begin{equation}
\Psi_B(s_1,s_2,\cdots,s_n)\left[\vec{q}_1,\cdots,\vec{q}_n\right] = N \sum_{\pi\in S_n}\bigotimes_{i = 1}^{n} \psi_{s_{\pi(i)}}\left(\vec{q}_i\right)
\end{equation}\end{linenomath}
以及
\begin{linenomath}\begin{equation}
\Psi_F(s_1,s_2,\cdots,s_n)\left[\vec{q}_1,\cdots,\vec{q}_n\right] = N' \sum_{\pi\in S_n}(-1)^{\pi}\bigotimes_{i = 1}^{n} \psi_{s_{\pi(i)}}\left(\vec{q}_i\right)
\end{equation}\end{linenomath}
其中 $(-1)^{\pi}$ 是置换 $\pi$ 的符号,$N$ 和 $N'$ 是归一化常数,他们的值将在稍后确定。但是,显而易见的是现在的波函数的确是(反)对称的。他们的正交性可以验证,只需要注意到一次量子化的波函数具有这样的正交归一性:
\[
    \begin{aligned}
        asd & schrodinger
        ad &
    \end{aligned}
\]



