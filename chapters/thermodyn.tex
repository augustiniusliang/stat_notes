\chapter{宏观热力学量与 Gibbs 分布}
“宏观\textbf{热力学量}” 事实上是一个同语反复(\textit{*tautologos}),因为只有宏观量才能为我们所测量。本章的目标是:介绍常见的热力学量,并指出他们与 Gibbs 分布的参量之间的关系。我们已经知道了更一般的情况,就是粒子数可以变化(由化学势支配)的情况,让我们把这些情况也考虑在内。
\section{内能}
我们已经知道,在系统与外界不发生宏观的相互作用时,我们可以写出下式:
\[
dE = -TdS + \sum_i\mu_idN_i.
\]
其中已经计及多种粒子所具有的不同化学势。这个式子的物理意义是:当没有宏观的相互作用时,物体与外界仅可通过传热和交换粒子传递相互作用。现在想要知道当外界条件发生变化时,物体的内能将会发生怎样的变化。
\subsection{准静态过程、可逆过程与元功}
为了不破坏平衡态,我们认为引入相互作用的过程是如此的慢,以至于它在力学上可以使用准静态过程(对于力学量 $\lambda$ 来说,这要求 $\frac{d\lambda}{dt}\ll 1$ )描述。但是若我们在引入相互作用时使熵 $S$ 和粒子数 $N$ 发生了改变,那么内能的形式将会难以写出。所以在这里我们只研究粒子数不变的可逆过程( $S$ 不变)。

在引入相互作用时,我们可以将熵随时间的变化 $\partial_t S$ 展成力学参量变化 $\partial_t \lambda$ 的级数,并假定这是引起熵变化的唯一来源(稍后将解释其中原因)。 由于平衡态的假设和熵增原理,我们可以知道领头阶的项是二阶的。于是:
\[
a
\]
